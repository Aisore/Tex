\documentclass[twocolumn,a4paper]{article}
\usepackage[T2A]{fontenc}
\usepackage[utf8]{inputenc}
\usepackage[english,russian]{babel}
\usepackage{amssymb,amsfonts,amsmath,mathtext}
\usepackage[landscape,margin=2cm]{geometry}
\geometry{left=1.5cm}
\geometry{right=1.5cm}

\begin{document}
%void
\sf{\slshape{Несобственные кратные интегралы}}

полного взаимно однозначного соответствия между множествами $n$ чисел $(x_1,\ldots,x_n)$ и $(\rho,\phi_1,\ldots,\phi_{n-1})$.

Отметим, что
\begin{center}
$\rho = \sqrt{x_1^2+\cdots+x_n^2}$.
\end{center}
Элементарными, но несколько громоздкими вычислениями, которые не будем здесь приводить, можно показать, что якобиан этого преобразования имеет вид
\begin{center}
$\frac{\partial(x_1,x_2,\ldots,x_n)}{\partial(\rho,\phi_1,\ldots,\phi_{n-1})} = \rho^{n-1}\cos{\phi_2}\cos^2{\phi_3}\ldots\cos^{n-2}{\phi_{n-1}}$.
\end{center}
Положим для краткости
\begin{center}
$\Phi(\phi_2,\ldots,\phi_{n-1}) = \cos{\phi_2}\cos^2{\phi_3\ldots\cos^{n-2}{\phi_{n-1}}}$.
\end{center}
Легко убедиться, что
\begin{center}
$c = \int\limits_0^{2\pi} \int\limits_{-\frac{\pi}{2}}^{\frac{\pi}{2}}\ldots\int\limits_{-\frac{\pi}{2}}^{\frac{\pi}{2}} \Phi(\phi_2,\ldots,\phi_{n-1})d\phi_1\ldots d\phi_{n-1} > 0$\footnote[1]{% 
Это следует из того факта, что если в кубируемой области $G$ функция $f$ непрерывна и неотрицательа и если существует точка $x^{(0)}\in G$ такая, что $f(x^{(0)}) > 0$, то $\int dG > 0$. Действительно, выберем какое-либо $\eta > 0$ так, чтобы $f(x^{(0)}) > \eta > 0$  Тогда существует $\delta > 0$ такая, что для всех $x \in O = O(x^{(0)}, \delta)$ выполняется неравенство $f(x) > \eta$ и $\int fdG \geq \int fdO \geq \eto mes\ O > O$.}
\end{center}

Исследуем теперь сходимость интеграла (48.7). В качестве последовательности кубируемых множеств $G_k, k = 1, 2,\ldots$,
монотонно исчерпывающей внешность единичного шара $Q$, возьмем последовательность множеств
\begin{center}
$G_k = \{\,x = (\rho,\phi_1,\ldots,\phi_{n-1}):1 + \frac1k < \rho < k\,\}, k = 1,2,\ldots$.
\end{center}
Перейдем к сферическим координатам:
\begin{center}
$\int \underset{G_k}{\ldots} \int\frac{dx_1\ldots dx_n}{(\sqrt{x^2_1+\ldots +x^2_n})^a} = $
\end{center}
\begin{center}
$\int\limits^k_{1 + \frac1k}\int\limits_0^{2\pi}\int\limits_{-\frac{\pi}{2}}^{\frac{\pi}{2}}\cdot \cdot \int\limits_{-\frac{\pi}{2}}^{\frac{\pi}{2}}\rho^{n-1-\alpha}\Phi(\phi_2,\ldots,\phi_{n-1})d\rho d\phi_1\ldots d\phi_{n-1}$
\end{center}

\newpage

{\slshape{48.3 Несобственные интегралы функций, имеющих знак}}
\begin{center}
$= c \int\limits^k_{1 + \frac1k} \rho^{n-1-\alpha}d\rho$.
\end{center}
Таким образом, вопрос о сходимости интеграла (48.7) свелся к сходимости интеграла $\int\limits_1^{\infty}\rho^{n-1-\alpha}d\rho$, который, как известно (см.п.34.3), сходится при $n - 1 - \alpha < -1$, т.е. при $\alpha < n$, и расходится при $\alpha\le n$. Итак, доказана следующая лемма.

{\slshape{{\bfseries{Лемма 1.}} Интеграл (48.7) сходится, если %void
больше размерности пространства, и расходится в противном случае.}}

Рассмотрим теперь интеграл (48.8). Полагая
\begin{center}
$G_k = \{\,x = (\rho,\phi_1,\ldots,\phi_{n-1}):\frac1k < \rho < 1 - \frac1k\,\}, k = 3,4,\ldots$,
\end{center}
получим
\begin{center}
$\int \ldots \int\frac{dx_1\ldots dx_n}{(\sqrt{x^2_1+\ldots +x^2_n})^a} = $
\end{center}
\begin{center}
$\int\limits^{1-\frac1k}_{\frac1k}\int\limits_0^{2\pi}\int\limits_{-\frac{\pi}{2}}^{\frac{\pi}{2}}\cdot \cdot \int\limits_{-\frac{\pi}{2}}^{\frac{\pi}{2}}\rho^{n-1-\alpha}\Phi(\phi_2,\ldots,\phi_{n-1})d\rho d\phi_1\ldots d\phi_{n-1}$
\end{center}
\begin{center}
$= c \int\limits^{1-\frac1k}_{\frac1k} \rho^{n-1-\alpha}d\rho$.
\end{center}
Таким образом, вопрос о сходимости интеграла (48.8) свелся к сходимости интеграла $\int\limits_0^1 \rho^{n-1-\alpha}d\rho$. Этот интеграл, как известно, сходится, если $n - 1 - \alpha > -1$, т.е. если $\alpha < n$, и расходится в противном случае. Полученный результат сформулируем снова в виде леммы.

{\slshape{{\bfseries{Лемма 2.}} Интеграл (48.8) сходится, если $\alpha$
меньше размерности пространства, и расходится в противном случае.}}

Подобно одномерному случаю (см.п. 33.3 и п. 34.3) с помощью интегралов (48.7) и (48.8) можно сформулировать 
критерии сходимости несобственных кратных интегралов, однако мне не будем на этом подробно останавливаться.


\begin{flushleft}
{\bfseries  48.3 Несобственные интегралы от функций,\\
меняющих знак} %VOID

{\slshape{\bf{Определение 3.}} Несобственный интеграл $\int fdG $ называется абсолютно сходящимся, если сходится интеграл $\int |f|dG $.}
\end{flushleft}

\end{document}
