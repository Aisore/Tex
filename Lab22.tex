\documentclass[a4paper,12pt]{book}
\usepackage[T2A]{fontenc}
\usepackage[koi8-r]{inputenc}
\usepackage[english,russian]{babel}
\usepackage{amssymb,amsfonts,amsmath,mathtext,cite,enumerate,float} 

\usepackage{geometry} % Меняем поля страницы
\geometry{left=1.5cm}% левое поле
\geometry{right=1.5cm}% правое поле
\geometry{top=1cm}% верхнее поле
\geometry{bottom=2cm}% нижнее поле

\begin{document}
%void
\slshape {Несобственные кратные интегралы}
Полного взаимно однозначного соответствия между множествами $n$ чисел $(x_1,...,x_n)$
и %void
.

Отметим, что

\begin(center)
	$p = \sqrt{x_1^2+...+x_n^2}$.
\end(center)

Элементарными, но несколько громоздкими вычислениями, которые не будем здесь приводить, можно показать, что якобиан этого преобразования имеет вид

%void

Положим для краткости

%void

Легко убедиться, что

%void

Исследуем теперь сходимость интеграла (48.7). В качестве последовательности кубируемых множеств $G_k, k = 1, 2, ...$,
монотонно исчерпывающей внешность единичного шара $Q$, возьмем последовательность множеств

%void

Перейдем к сферическим координатам:

%void

\slshape {48.3 Несобственные интегралы функций, имеющих знак}

%void

Таким образом, вопрос о сходимости интеграла (48.7) свелся к сходимости интеграла %void
, который, как известно (см.п.34.3), сходится при $n -- 1 -- a < --1$, т.е. при $a < n$, и расходится при $a\le n$. Итак, доказана следующая лемма.

\slshape {{\bfseries {Лемма 1.}} Интеграл (48.7) сходится, если %void
больше размерности пространства, и расходится в противном случае.}

Рассмотрим теперь интеграл (48.8). Полагая

%void

получим

%void

Таким образом, вопрос о сходимости интеграла (48.8) свелся к сходимости интеграла %void
. Этот интеграл, как известно, сходится, если $n -- 1 -- a > --1$, т.е. если $a < n$, и расходится в противном случае. Полученный результат сформулируем снова в виде леммы.

\slshape {{\bfseries {Лемма 2.}} Интеграл (48.8) сходится, если $a$
меньше размерности пространства, и расходится в противном случае.}

Подобно одномерному случаю (см.п. 33.3 и п. 34.3) с помощью интегралов (48.7) и (48.8) можно сформулировать 
критерии сходимости несобственных кратных интегралов, однако мне не будем на этом подробно останавливаться.

\begin{center}
\bfseries {48.3 Несобственные интегралы от функций, меняющих знак}
\end{center}

\slshape {{\bfseries {Определение 3.}} Несобственный интеграл %void
называется абсолютно сходящимся, если сходится интеграл %void
.}
\end{document}
