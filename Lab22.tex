\documentclass[twocolumn]{article}
\usepackage[T2A]{fontenc}
\usepackage[utf8]{inputenc}
\usepackage[english,russian]{babel}
\usepackage{amssymb,amsfonts,amsmath,mathtext}
\usepackage[landscape,margin=2cm]{geometry}
\geometry{left=1.5cm}
\geometry{right=1.5cm}
\pagestyle{empty}
\begin{document}

\makeatletter
\renewcommand{\@oddhead}{\textrm{\footnotesize\bf{154}}{\footnotesize\slshape{\hspace{115pt}\S\ 48. \it Несобственные кратные интегралы}}\hfil\textrm{\footnotesize\slshape{\hspace{10pt} 48.3. \it Несобственные интегралы функций, меняющих знак \hspace{60pt}}}{\footnotesize\bf{155 \hspace{40pt}}}}
\makeatother

{\raggedright полного взаимно однозначного соответствия между множествами

$n$ чисел $(x_1,\ldots,x_n)$ и $(\rho,\phi_1,\ldots,\phi_{n-1})$.
}

{\raggedright Отметим, что}
\begin{center}
$\rho = \sqrt{x_1^2+\cdots+x_n^2}$.
\end{center}
\hspace{5pt} Элементарными, но несколько громоздкими вычислениями,\\ которые не будем здесь приводить, можно показать, что якобиан\\ этого преобразования имеет вид
\begin{center}
$\frac{\partial(x_1,x_2,\ldots,x_n)}{\partial(\rho,\phi_1,\ldots,\phi_{n-1})} = \rho^{n-1}\cos{\phi_2}\cos^2{\phi_3}\ldots\cos^{n-2}{\phi_{n-1}}$.
\end{center}
\hspace{5pt} Положим для краткости
\begin{center}
$\Phi(\phi_2,\ldots,\phi_{n-1}) = \cos{\phi_2}\cos^2{\phi_3\ldots\cos^{n-2}{\phi_{n-1}}}$.
\end{center}
\hspace{5pt} Легко убедиться, что
\begin{center}
$c = \int\limits_0^{2\pi} \int\limits_{-\frac{\pi}{2}}^{\frac{\pi}{2}}\ldots\int\limits_{-\frac{\pi}{2}}^{\frac{\pi}{2}} \Phi(\phi_2,\ldots,\phi_{n-1})d\phi_1\ldots d\phi_{n-1} > 0$\renewcommand{\thefootnote}{*)}\footnote{% 
Это следует из того факта, что если в кубируемой области $G$ фун-\\кция $f$ непрерывна и неотрицательна и если существует точка $x^{(0)}\in G$ та-\\кая, что $f(x^{(0)}) > 0$, то $\int dG > 0$. Действительно, выберем какое-либо\\ $\eta > 0$ так, чтобы $f(x^{(0)}) > \eta > 0$  Тогда существует $\delta > 0$ такая, что для\\ всех $x \in O = O(x^{(0)}, \delta)$ выполняется неравенство $f(x) > \eta$ и $\int fdG \geq \int fdO \\ \geq \eta\ mes\ O > 0$.}
\end{center}

\ Исследуем теперь сходимость интеграла (48.7). В качестве после-\\довательности кубируемых множеств $G_k,\ k = 1, 2,\ldots$,
монотонно\\ исчерпывающей внешность единичного шара $Q$, возьмем последо-\\вательность множеств
\begin{center}
$G_k = \{\,x = (\rho,\phi_1,\ldots,\phi_{n-1}):1 + \frac1k < \rho < k\,\}, k = 1,2,\ldots$.
\end{center}
\hspace{5pt} Перейдем к сферическим координатам:
\begin{center}
$\int \underset{G_k}{\ldots} \int\frac{dx_1\ldots dx_n}{(\sqrt{x^2_1+\ldots +x^2_n})^a} = $
\end{center}
\begin{center}
$\int\limits^k_{1 + \frac1k}\int\limits_0^{2\pi}\int\limits_{-\frac{\pi}{2}}^{\frac{\pi}{2}}\cdot \cdot \int\limits_{-\frac{\pi}{2}}^{\frac{\pi}{2}}\rho^{n-1-\alpha}\Phi(\phi_2,\ldots,\phi_{n-1})d\rho d\phi_1\ldots d\phi_{n-1}$
\end{center}

\newpage

\begin{center}
$= c \int\limits^k_{1 + \frac1k} \rho^{n-1-\alpha}d\rho$.
\end{center}
\hspace{5pt} Таким образом, вопрос о сходимости интеграла (48.7) свелся\\ к сходимости интеграла $\int\limits_1^{\infty}\rho^{n-1-\alpha}d\rho$, который, как известно (см.\\п.34.3), сходится при $n - 1 - \alpha < -1$, т.е. при $\alpha < n$, и расхо-\\дится при $\alpha\le n$. Итак, доказана следующая лемма.

{\slshape {\bfseries{Лемма 1.\ }}\it Интеграл (48.7) сходится, если $\alpha$ больше размерно-\\сти пространства, и расходится в противном случае.}

\hspace{5pt} Рассмотрим теперь интеграл (48.8). Полагая
\begin{center}
$G_k = \{\,x = (\rho,\phi_1,\ldots,\phi_{n-1}):\frac1k < \rho < 1 - \frac1k\,\}, k = 3,4,\ldots$,
\end{center}
получим
\begin{center}
$\int \ldots \int\frac{dx_1\ldots dx_n}{(\sqrt{x^2_1+\ldots +x^2_n})^a} = $
\end{center}
\begin{center}
$= \int\limits^{1-\frac1k}_{\frac1k}\int\limits_0^{2\pi}\int\limits_{-\frac{\pi}{2}}^{\frac{\pi}{2}}\cdot \cdot \int\limits_{-\frac{\pi}{2}}^{\frac{\pi}{2}}\rho^{n-1-\alpha}\Phi(\phi_2,\ldots,\phi_{n-1})d\rho d\phi_1\ldots d\phi_{n-1} =$
\end{center}
\begin{center}
$= c \int\limits^{1-\frac1k}_{\frac1k} \rho^{n-1-\alpha}d\rho$.
\end{center}
\hspace{5pt} Таким образом, вопрос о сходимости интеграла (48.8) свелся\\ к сходимости интеграла $\int\limits_0^1 \rho^{n-1-\alpha}d\rho$. Этот интеграл, как известно,\\ сходится, если $n - 1 - \alpha > -1$, т.е. если $\alpha < n$, и расходится в\\ противном случае. Полученный результат сформулируем снова в\\ виде леммы.

{\slshape{{\bfseries\slshape{Лемма 2.\ }}    \it Интеграл (48.8) сходится, если $\alpha$
меньше размерно-\\сти пространства, и расходится в противном случае.}}

Подобно одномерному случаю (см.п. 33.3 и п. 34.3) с помощью\\ интегралов (48.7) и (48.8) можно сформулировать 
критерии сходи-\\мости несобственных кратных интегралов, однако мне не будем\\ на этом подробно останавливаться.


\begin{flushleft}
{\small{\bfseries\hspace{30pt} 48.3 Несобственные интегралы от функций,\\
\hspace{30pt} меняющих знак}}
\end{flushleft}
{\slshape{\bfseries\slshape{\hspace{30pt} Определение 3.\ }}\it Несобственный интеграл $\int fdG $ назы-\\вается абсолютно сходящимся, если сходится интеграл $\int |f|dG $.}


\end{document}
